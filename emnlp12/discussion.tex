\section{Error Analysis}
\label{sec:discuss}

Figure~\ref{plot-hinton} is the Hinton diagram showing the
relationship between the most frequent tags and clusters from the
experiment in Section~\ref{sec:feat}.  In general the errors seem to
be the lack of completeness (multiple large entries in a row), rather
than lack of homogeneity (multiple large entries in a column).  The
algorithm tends to split large word classes into several clusters.
Some examples are:
\begin{itemize}
\item Titles like Mr., Mrs., and Dr. are split from the rest of the
  proper nouns in cluster (39).
\item Auxiliary verbs (10) and the verb ``say'' (22) have been split
  from the general verb clusters (12) and (7).
\item Determiners ``the'' (40), ``a'' (15), and capitalized
  ``The'', ``A'' (6) have been split into their own clusters.
\item Prepositions ``of'' (19), and ``by'', ``at'' (17) have been
  split from the general preposition cluster (8).
\end{itemize}
Nevertheless there are some homogeneity errors as well:
\begin{itemize} 
\item The adjective cluster (5) also has some noun members probably
  due to the difficulty of separating noun-noun compounds from
  adjective modification.
\item Cluster (6) contains capitalized words that span a number of
  categories.
\end{itemize}

Most closed-class items are cleanly separated into their own clusters
as seen in the lower right hand corner of the diagram.  The
completeness errors are not surprising given that the words that have
been split are not generally substitutable with the other members of
their Penn Treebank category.  Thus it can be argued that metrics that
emphasize homogeneity such as \mto are more appropriate in this
context than metrics that average homogeneity and completeness such as
\vm as long as the number of clusters is controlled.

